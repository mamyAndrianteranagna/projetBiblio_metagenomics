\section{Introduction}

	\begin{itemize}
		
        \item[•] définition et buts de la métagénomique
          \begin{itemize}
          \item métagénomique = étude des populations microbiennes via le métagénome
          \item métagénome = ensemble des génomes de tous les espèces présentes dans un environnement donné
          \item métagénomique $\rightarrow$ identification, classification et quantification des espèces microbiennes présentes dans un échantillon d'un milieu donné
          \item métagénomique $\rightarrow$ exploration des populations microbiennes des océans, du sol, des tubes digestifs, etc.
          \item métagénomique $\rightarrow$ étude des espèces microbiennes non cultivables
          \end{itemize}
          
        \item[•] historique et évolution de la métagénomique \cite{Handelsman2004a}
          \begin{itemize}
          \item milieux des années 80 : prise de conscience des microbiologistes sur l'importance et le besoin d'étudier les microorganismes non cultivables -> classification (phylogénétique) des espèces présentes dans un milieux sauvage donné, grace à l'utilisation de plus en plus facile des séquences d'ARNr 16S (phylogenetic stain) rendue facile (cette facilité est due à des progrès techniques telles que publié dans \cite{lane1985})
          \item début des années 90: PCR $\rightarrow$ possibilité de cloner entièrement le gène d'ARNr 16S à partir du milieu sans passer par des techniques lourdes $\Rightarrow$ rapidité et efficacité de la détermination et classification des nouvelles espèces microbiennes \cite{Schmidt1991}
          \item fin des années 90 : naissance de l'appellation métagénomique \cite{Handelsman1998}
          \item au début, la métagénomique sert uniquement à identifier les espèces présentes dans le milieu étudié puis, par la suite, elle permet aussi de caractériser leurs fonctions
          \end{itemize}
          
        \item[•] approches de la métagénomique
          \begin{itemize}
          \item approche par séquençage (sequence-based analysis): séquençage des marqueurs de phylogénétiques (ex. ARNr16S) ou shotgun sequencing
          \item approche fonctionnelle (functionnal métagénomics) $\rightarrow$ hétérologous expression (\textit{E.coli})
          \end{itemize}
          
        \item[•] la nouvelle génération de métagénomique
          \begin{itemize}
          \item influence de l'apparition de NGS $\Rightarrow$ shotgun sequencing metagenomics
          \item objectif de la classification: comparaison des frangments de séquences à une séquence génomique de référence
          \item méthodes de classifications: amplicon-based et shotgun sequencing based \cite{Mande2012}, taxonomy dependent ($\rightarrow$ classification supervisée $\rightarrow$ utilisation de références phylogénétiquement connues) or taxonomy independent ($\rightarrow$ classification non supervisée)
          \item méthode de classification la plus répandue pour shotgun sequencing metagenomics $\rightarrow$ comptage des k-mers communs (entre les reads et chaque génome de référence) $\rightarrow$ classification
          \end{itemize}

        \item[•] objectif de l'étude
          \begin{itemize}
            \item proposition de \cite{Brinda2015} d'utiliser les k-mers espacés à la place des k-mers contigus pour améliorer la sensibilité/spécificité de la classification
            \item objectif: comparaison des résultats de classification (sensibilité/spécificité), comparaison sur l'implémentation et la complexité
          \end{itemize}
	\end{itemize}
