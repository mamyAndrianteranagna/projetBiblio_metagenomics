\section{Introduction}

	\begin{itemize}
		\item[•] définition et buts de la métagénomique
		  \begin{itemize}
			\item métagénomique = étude des populations microbiennes via le métagénome
			\item métagénome = ensemble des génomes de tous les espèces présentes dans un environnement donné
			\item métagénomique \rightarrow identification, classification et quantification des espèces microbiennes présentes dans un échantillon d'un milieu donné
                        \item métagénomique \rightarrow exploration des populations microbiennes des océans, du sol, des tubes digestifs, etc.
                        \item métagénomique \rightarrow étude des espèces microbiennes non cultivables
		  \end{itemize}
		\item[•] objectifs et principe de la métagénomique
                  \begin{itemize}
                  \item purification de l'ADN génomique du milieu
                  \item digestion de l'ADN génomique
                  \item sous-clonage de chaque fragment dans de bacteriophage
                  \item séquençage
                  \item comparaison de séquences aux séquences présentes dans les bases de données
                  \end{itemize}
                \item[•] historique et évolution de la métagénomique \cite{Handelsman2004a}
                  \begin{itemize}
                  \item milieux des années 80 : prise de conscience des microbiologistes sur l'importance et le besoin d'étudier les microorganismes non cultivables -> classification (phylogénétique) des espèces présentes dans un milieux sauvage donné, grace à l'utilisation de plus en plus facile des séquences d'ARNr 16S (phylogenetic stain) rendue facile (cette facilité est due à des progrès techniques telles que publié dans \cite{lane1985})
                  \item début des années 90: PCR \rightarrow possibilité de cloner entièrement le gène d'ARNr 16S à partir du milieu sans passer par des techniques lourdes \Rightarrow rapidité et efficacité de la détermination et classification des nouvelles espèces microbiennes \cite{Schmidt1991}
                    \item fin des années 90 : naissance de l'appellation métagénomique \cite{Handelsman1998}
                  \item au début, la métagénomique sert uniquement à identifier les espèces présentes dans le milieu étudié, actuellement elle permet aussi de caractériser leurs fonctions
                    
                  \end{itemize}
		\item[•] principe de la nouvelle génération de métagénomique
	\end{itemize}
