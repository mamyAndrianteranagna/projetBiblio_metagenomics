\section{Discussions et Conclusion}


\subsection{L'utilisation des k-mers espacés represente-t-elle rééllement des intérêts à la métagénomique actuelle?}

Juste avant l'apparition de la NGS, un projet de métagénomique compte environ 2 millions de reads (1 milliard de bp). Après l'apparition de la NGS, ce nombre devient de l'ordre de milliard de reads \cite{Kumar2015} (produit en un seul run de plateforme NGS). Ces énormes données produites par la NGS imposent alors depuis un grand défit sur le temps de calcul et l'espace mémoire lors de leur traitement.

En outre, les courtes séquences des reads produites par le NGS (par rapport au séquençage Sanger utilisé auparavant) apporte une grande difficulté d'analyse notamment pour l'assemblage de novo \cite{Kumar2015}.
Face à ces défis qui attends encore la métagénomique actuelle, la question se pose si l'amélioration de la sensibilité apportée par la méthode des k-mers espacés est-elle primordiale. Etant donné que cette amélioration est aussi accompagné de coût additionnel en temps et en espace, la réponse semble plutôt négative. Ce qui ne veut pas dire que ce travail inutile, il est certe intéressant mais devrait être accompagné par des amélioration au niveau facilité de calcul et de stockage. D'ailleurs, beaucoup d'autres travaux vont dans la même direction en essayant d'améliorer la performance de la classification métagénomique.  


\subsection{Les autres tentatives d'amélioration de la classification métagénomique}

Le nombre de travaux effectués sur l'algorithme de classification métagénomique montre l'intérêt dans un futur proche de l'amélioration de cette performance de la classification métagénomique.
Parmi ces travaux, on peut citer celui de Kumar et al. \cite{Kumar2015} et de Wang et al. \cite{Wang2015}. 
Ce dernier propose le text mining appliqué dans la classification métagénomique.
Ramazotti et al. propose aussi une nouvelle méthode de métagénomique combinant la classification non dirigée et la classification dirigée pour améliorer la classification à taxonomie indépendante c'est à dire sans utilisation de génomes de références.

%\centering
%***

%\newpage{}