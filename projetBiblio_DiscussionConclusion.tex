\section{Discussions et Conclusion}

	\begin{itemize}
		\item[•] intérêt des deux méthodes aux contextes actuels de l'étude
                  \begin{itemize}
                  \item facteur limitant (bottleneck) pour la métagénomique actuelle: juste avant l'apparition de la NGS, les reads d'un projet métagénome est de l'ordre de 2 millions de reads (1 milliard de bp), après l'apparition de la NGS, le nombre de reads produits par un seul projet est de l'ordre de milliard \cite{Kumar2015} $\rightarrow$ ces énormes données produites par la NGS demanderai plutôt de traitement bioinformatique à moindre coût en temps et en espace, reads de NGS de courte séquence par rapport au Sanger $\Rightarrow$ difficulté d'analyse notamment pour l'assemblage de novo, \cite{Kumar2015} rélate plus les limites au niveau de l'assemblage de novo des métagénomes
                  \item l'amélioration de la sensibilité apportée par la méthode des k-mers espacés est-elle primordiale pour surmonter ces limites? Etant donnée que celle-ci comporte des coût additionnels en temps et espace?
                  \end{itemize}
		\item[•] perspective (intérêt futur de la nouvelle approche?)
                  \begin{itemize}                
                  \item la méthode des k-mers présentent-elle un intérêt majeur pour la classification métagénomique dans le futur proche?
                  \item les autres tentatives/pistes d'amélioration de la classification métagénomique: développement continuel des algo pour la métagénomique \cite{Kumar2015}, text mining et métagénomique \cite{Wang2015}
                  \item \cite{Ramazotti2015} propose une nouvelle méthode combinant le shotgun sequencing et l'amplicon-based metagenomic pour la classification non supervisée (taxonomy independant)
                  \end{itemize}
	\end{itemize}