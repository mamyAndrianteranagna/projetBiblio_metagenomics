\section{Les algorithmes de classification des reads}

	\begin{itemize}
		\item[•] méthode des k-mers contigus dans la classification métagénomique
                  \begin{itemize}
                  \item construction des k-mers de longueur l sur un read de longueur L, décalé à la fois d'une seule base
                  \item nb de k-mers = L-l+1
                  \end{itemize}
		\item[•] méthode des k-mers espacées
                  \begin{itemize}
                  \item concept venant directe des graines espacées des alignements par extension de graine
                  \item graines espacées: proposé pour la première fois par \cite{Ma2002} dans son algo PatternHunter pour garder la rapidité de comparaison lors des alignements par graines (cette rapidité diminue dû à l'augmentation des tailles des bases de données) tout en augmentant la sensibilité
                  \item nombre de hit: nb de k-mers (espacées) retrouvés dans le génome de référence
                  \item couverture: nb total de positions couvertes par tous les k-mers matchés
                  \item proposés par différents auteurs pour améliorer différentes méthodes d'alignement de reads dans différents contexte d'analyse NGS: Chip-Seq \cite{Ghandi2014}, reconstruction phylogénétique \cite{Leimeister2014}, 
                  \end{itemize} 
	\end{itemize}